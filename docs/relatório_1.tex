\documentclass{article}

\usepackage{multicol}
\usepackage[T1]{fontenc}        % Codificação para português (ao copiar e colar do documento, acentos são colados corretamente)
\usepackage[portuguese]{babel}  % Português (alguns pacotes usam termos da linguagem, por ex 'Referências')

\author{Igor Lacerda}
\title{Trabalho de Implementação - Parte I}
\date{Heurísticas e Meta-heurísticas}

\begin{document}

\maketitle

O Trabalho de Implementação de Heurísticas consiste na implementação de heurísticas para o Problema do Caixeiro Viajante (TSP). Na primeira parte, era necessário escolher e implementar alguma heurística construtiva. Foi escolhida a heurística \textit{Twice Around The Tree} (TATT).

A ideia por trás da TATT é bem simples: embora seja complicado encontrar uma solução para o TSP, é muito fácil encontrar uma árvore geradora mínima, que é um problema parecido. Dado um grafo, qual das árvores que cobre todos os vértices tem o menor custo? Existem vários algoritmos (por exemplo, Kruskal e Prim) que respondem essa pergunta de maneira eficiente.

É intuitivo pensar que a árvore que percorre todos os vértices, de maneira a minimizar o custo, deve compartilhar algumas arestas com o ciclo hamiltoniano mínimo (TSP). Mas é preciso transformar a árvore em um caminho, e depois ``fechá-lo''. Na TATT isso é feito através do caminhamento em profundidade (pré-ordem), com o vértice inicial adicionado no final.

\section*{Resultados}

\begin{multicols}{2}
    \begin{tabular}{c c c}
        Nome & Tempo (s) & Custo \\
        \hline
        kroE100 & 0.0064 & 30507.41 \\
        kroC100 & 0.0063 & 27966.54 \\
        kroB100 & 0.0061 & 25881.19 \\
        kroD100 & 0.0062 & 27113.29 \\
        berlin52 & 0.0017 & 10116.01 \\
        pr76 & 0.0039 & 145338.11 \\
        pr144 & 0.0146 & 80596.32 \\
        rat195 & 0.0608 & 3317.72 \\
        kroA150 & 0.0185 & 35122.56 \\
        kroA100 & 0.0070 & 27211.67 \\
        \hline
    \end{tabular}
    \begin{tabular}{c c c}
        Nome & Tempo(s) & Custo \\
        \hline
        pr136 & 0.0129 & 151913.74 \\
        kroB200 & 0.0306 & 40710.95 \\
        pr152 & 0.0173 & 87998.69 \\
        kroA200 & 0.0349 & 40030.86 \\
        pr107 & 0.0082 & 54238.03 \\
        st70 & 0.0035 & 873.35 \\
        pr124 & 0.0545 & 74140.95 \\
        rat99 & 0.0067 & 1723.22 \\
        kroB150 & 0.0185 & 36154.73 \\
        lin105 & 0.0072 & 19498.40 \\
        \hline
    \end{tabular}
\end{multicols}

\end{document}
